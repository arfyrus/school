\documentclass[a4paper, 12pt]{article}
\usepackage{graphicx, fancyhdr, geometry, mathptmx, titlesec, setspace}

\pagestyle{fancy}
\fancyhead{}
\fancyfoot[C]{\thepage}
\newcommand{\Besar}{\fontsize{14pt}{1.5}\selectfont}
\renewcommand{\headrulewidth}{0pt}
\renewcommand{\contentsname}{\centering\Besar ISI KANDUNGAN}
\geometry{margin=1in}
\onehalfspacing

\titleformat{\section}[block]{\normalfont\centering\Besar\bfseries}{}{0em}{}
\titleformat{\subsection}[block]{\normalfont\Besar\bfseries}{\thesubsection}{1em}{}

\title{Fasa Test}
\author{MUHAMMAD ARIF BIN MOHD HAFEEZ Moe}
\date{December 2024}

\begin{document}

\addcontentsline{toc}{section}{PENGHARGAAN}
\section*{PENGHARGAAN}
\pagenumbering{Roman}

\begin{center}
    Saya ingin merakamkan ribuan terima kasih kepada guru mata pelajaran Sains Komputer 
\end{center}

\newpage
\tableofcontents

\newpage
\section{FASA 1 \\[1em] ANALISIS MASALAH}
\pagenumbering{arabic}

\subsection{Peryataan Masalah}

Restoran “Hiroshima Sushi” merupakan sebuah restoran yang menjual pelbagai jenis makanan, terutamanya makanan laut. Apabila pelanggan hendak memesan makanan, dia akan memberi pesanan makanan kepada pelayan. Pelayan pula menulis pesanannya pada kertas dan memberikannya kepada tukang masak. Oleh demikian, beberapa masalah mungkin timbul dalam restoran ini. Contohnya, tukang masak silap membaca pesanan pelanggan akibat tulisan yang tidak jelas.

\subsection{Objektif}

\begin{itemize}
    \item Merekod maklumat pelanggan dan pesanan supaya boleh dipesan semula lain kali mereka datang.
    \item Memaparkan maklumat pesanan pelanggan dan boleh dikemaskini.
    \item Memaparkan makanan dan minuman yang dijual dan yang tidak tersedia.
\end{itemize}

\subsection{Skop}

\begin{itemize}
    \item Sistem ini hanya menunjuk makanan dan minuman yang ditawarkan di Restoran “Hiroshima Sushi”.
    \item Sistem ini tidak memberi pilihan kepada pelanggan untuk membayar secara dalam talian.
\end{itemize}

\subsection{Kumpulan Sasaran}

\begin{itemize}
    \item Pelanggan Restoran “Hiroshima Sushi”.
    \item Pengurus Restoran “Hiroshima Sushi”.
\end{itemize}

\subsection{Menilai Sistem Sedia Ada}

\subsubsection*{Fungsi Sistem Sedia Ada}

\begin{itemize}
    \item Sistem sedia ada ialah pelanggan memberi pesanan makanan kepada pelayan. Pelayan menulis pesanan pelanggan pada kertas dan memberikannya kepada tukang masak.
\end{itemize}

\subsubsection*{Kekuatan}

\begin{itemize}
    \item Sistem ini membolehkan pelanggan meminta penjelasan tentang makanan dan minuman yang ingin dipesan.
    \item Tidak memerlukan kos yang tinggi kerana hanya menggunakan pensil dan kertas.
    \item Pelanggan tidak perlu mempunyai ilmu penggunaan komputer untuk membuat pesanan.
\end{itemize}

\subsubsection*{Kelemahan}

\begin{itemize}
    \item Pelanggan harus menunggu masa yang lama kerana pelayan melayan pelanggan yang ramai. 
    \item Pelayan mungkin salah dengar pesanan pelanggan.
    \item Kebarangkalian terdapat maklumat tempahan yang bertindih.
\end{itemize}

\subsubsection*{Justifikasi Sistem Dibangunkan}

\begin{itemize}
    \item Membantu mencari maklumat pelanggan.
    \item Merekodkan tempahan dengan efisyen.
    \item Memaparkan tempahan mengikut pelanggan atau tarikh.
\end{itemize}

\subsubsection*{Nama System}

\paragraph{} System pesanan makanan "Shima Pesan".

 \end{table}

\end{document}
